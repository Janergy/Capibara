\usepackage{booktabs}
\usepackage{blindtext}
\usepackage{xargs}
\usepackage{nth}

%-----------%
% XREF DICT %
%-----------%

\usepackage{pgfkeys}
\usepackage[table]{xcolor}
\pgfkeyssetvalue{/xref dict/fig}{Figure}
\pgfkeyssetvalue{/xref dict/eq}{Equation}
\pgfkeyssetvalue{/xref dict/eqs}{Equations}
\pgfkeyssetvalue{/xref dict/tab}{Table}
\pgfkeyssetvalue{/xref dict/def}{Definition}
\pgfkeyssetvalue{/xref dict/note}{Note}
\pgfkeyssetvalue{/xref dict/challenge}{Challenge}
\pgfkeyssetvalue{/xref dict/example}{Example}
\pgfkeyssetvalue{/xref dict/0}{tapir_rainbow}
\pgfkeyssetvalue{/xref dict/1}{tapir_voronoi}
\pgfkeyssetvalue{/xref dict/2}{tapir_stars}
\pgfkeyssetvalue{/xref dict/3}{tapir_comics}
\pgfkeyssetvalue{/xref dict/4}{tapir_strawberry_two}
\pgfkeyssetvalue{/xref dict/5}{tapir_fourier}
\pgfkeyssetvalue{/xref dict/6}{tapir_warhol}
\pgfkeyssetvalue{/xref dict/7}{tapir_butterfly}

%----------%
% GRAPHICS %
%----------%

\usepackage{svg}
\usepackage{caption}
\usepackage{subcaption}

\usepackage{tikz}
\usepackage{tikz-3dplot}
\usepackage{tikzpagenodes}
\usetikzlibrary{backgrounds, math, positioning, calc, decorations.pathreplacing, decorations.text, decorations.markings, quotes, perspective, tikzmark, fpu, matrix, fit, shapes, shadows}

\makeatletter
\tikzoption{canvas is plane}[]{\@setOxy#1}
\def\@setOxy O(#1,#2,#3)x(#4,#5,#6)y(#7,#8,#9)%
  {\def\tikz@plane@origin{\pgfpointxyz{#1}{#2}{#3}}%
   \def\tikz@plane@x{\pgfpointxyz{#4}{#5}{#6}}%
   \def\tikz@plane@y{\pgfpointxyz{#7}{#8}{#9}}%
   \tikz@canvas@is@plane
  }
\makeatother

\tikzset{
  arrow/.style = {thick, ->, >=stealth},
  vector/.style = {ultra thick, ->, >=stealth, cap=round},
  complex/.style = {circle, minimum size=5pt, fill, inner sep=0pt},
  axisline/.style = {thick, stealth-stealth},
  mathl/.style = {fill=#1, rectangle, minimum width=10pt, minimum height=10pt},
  shnode/.style={black, midway, anchor=center},
  mat/.style={matrix of math nodes, nodes={rectangle, draw, minimum size=1cm, anchor=center}},
  boxarr/.style={->, >=stealth, line width=3pt},
  frame/.style={draw=#1, rounded corners, fill=#1!20},
  perp/.style={black!50, thick, dashed},
  component/.style = {line width=5pt, draw=#1},
  element/.style = {draw=black, anchor=west, circle, minimum size=7mm, fill=#1},
  arcnode/.style 2 args = {decorate, decoration={raise=#1, markings, mark=at position 0.5 with {\node[inner sep=0] {#2};}}},
  % Easier arcs
  % Taken from https://tex.stackexchange.com/questions/66490/drawing-a-tikz-arc-specifying-the-center
    pics/carc/.style args={#1:#2:#3}{
    code={
      \draw[pic actions] (#1:#3) arc (#1:#2:#3);
    }
  },
  % Draw perpendicular lines symbol
  % Taken from https://tex.stackexchange.com/questions/21752/insertion-of-perpendicular-symbol-at-intersection-of-two-perpendicular-lines
  right angle quadrant/.code={
    \pgfmathsetmacro\quadranta{{1,1,-1,-1}[#1-1]}     % Arrays for selecting quadrant
    \pgfmathsetmacro\quadrantb{{1,-1,-1,1}[#1-1]}},
  right angle quadrant=1, % Make sure it is set, even if not called explicitly
  right angle length/.code={\def\rightanglelength{#1}},   % Length of symbol
  right angle length=2ex, % Make sure it is set...
  right angle symbol/.style n args={3}{
    insert path={
        let \p0 = ($(#1)!(#3)!(#2)$) in     % Intersection
            let \p1 = ($(\p0)!\quadranta*\rightanglelength!(#3)$), % Point on base line
            \p2 = ($(\p0)!\quadrantb*\rightanglelength!(#2)$) in % Point on perpendicular line
            let \p3 = ($(\p1)+(\p2)-(\p0)$) in  % Corner point of symbol
        (\p1) -- (\p3) -- (\p2)
    }
  },
  declare function={
    zoomG(\x)=-(\x-2.4)^3+2*(\x-2.4)^2;
    zoomF(\x,\a)=zoomG(\x)-zoomG(\a)+\a;
    DzoomG(\x)=-(5*\x-12)*(15*\x-56)/25;
  },
  % Legendre polynomials
  % Shamelessly stolen from https://tex.stackexchange.com/questions/595780/recursive-function-in-pgfplots
  evaluate={
    function legendre(\x,\p) {
      if \p == 0 then {
        return 1;
      } else {
        return legendre2(\x, \p);
      };
    };
    function legendre2(\x, \p) {
      if \p == 1 then {
        return \x;
      } else {
        return (1/\p)*((2*\p-1)*\x*legendre(\x, \p-1) + (1-\p)*legendre(\x, \p-2));
      };
    };
    function legendre3(\n, \x){
      \S = 0;
      int \A; \A = 3;
      for \k in {1,...,\A}{
        \S=\A;
      };
      return \S;
    };
  },
}

\newcommand{\draworigin}{\fill (0,0) circle (0.075);}
\newcommand{\tikzinline}[1]{\tikz[baseline=-0.5ex]{#1}}
\newcommand{\tikzhl}[3]{\tikzinline{\node[fill=#1!20](#2){$#3$}}}

% Draw basis vector expansion for a 2D vector
% (use inside a TikZ env)
\newcommand{\basisvecs}[4]{
  % #1: 1st basis vector: a
  % #2: 2nd basis vector: b
  % #3: 1st basis component
  % #4: 2nd basis component
  \def\aint{\intpart{#3}}
  \pgfmathsetmacro{\ainteger}{abs(\aint)}
  \def\afrac{0.\fracpart{#3}}
  \pgfmathparse{ifthenelse(\aint>=0,+1.0,-1.0)}\pgfmathsetmacro{\asign}{\pgfmathresult}%
  \def\bint{\intpart{#4}}
  \def\bfrac{0.\fracpart{#4}}
  \pgfmathparse{ifthenelse(\bint>=0,+1.0,-1.0)}\pgfmathsetmacro{\bsign}{\pgfmathresult}%

  % \ainteger
  % 1st basis vector loop
}

\usepackage{pgfplots}
\usepgfplotslibrary{fillbetween, colormaps, colorbrewer, patchplots}
\usepgfmodule{nonlineartransformations}
\pgfplotsset{
		compat=1.16,
		every axis/.append style={
			font=\small,
		},
    graph2d/.style={
      axis x line=middle,
      axis y line=middle,
      every axis x label/.style={
        at={(ticklabel* cs:1.01)},
        anchor=west,
      },
      every axis y label/.style={
        at={(ticklabel* cs:1.01)},
        anchor=south,
      },
      axis line style={stealth-stealth, thick},
      label style={font=\large},
      xlabel=$x$,
      ylabel=$y$,
      tick label style={font=\small},
      samples=200,
      grid=both,
      grid style={line width=.1pt, draw=gray!20},
      major grid style={line width=.2pt,draw=gray!50},
      minor tick num=4,
    },
	function/.style={line width=1.5pt},
	vector plane no ticks/.style={
		vector plane,
		width=5cm, height=5cm,
		xmin=-4, xmax=4,
		ymin=-4, ymax=4,
		tick style={draw=none},
		xticklabels={,},
		yticklabels={,},
	},
	linear plane no ticks/.style={
		vector plane,
		width=5cm, height=5cm,
		xmin=-4, xmax=4,
		ymin=-4, ymax=4,
		tick style={draw=none},
		xticklabels={,},
		yticklabels={,},
		samples=2,
	},
	empty/.style={
		vector plane,
		axis line style={draw=none},
		tick style={draw=none},
		xlabel={},
		ylabel={},
		xticklabels={,},
		yticklabels={,},
		grid=none,
	},
	axes 3D/.style={
		axis lines=center,
		axis line style={stealth-stealth, thick},
		xlabel={$x$},
		ylabel={$y$},
		zlabel={$z$},
		every axis x label/.style={at={(ticklabel* cs:1.01)}, anchor=west},
		every axis y label/.style={at={(ticklabel* cs:1.01)}, anchor=west},
		every axis z label/.style={at={(ticklabel* cs:1.01)}, anchor=south},
		z buffer=sort,
		xmin=-4, xmax=4,
		ymin=-4, ymax=4,
		zmin=-4, zmax=4,
		xtick=\empty,
		ytick=\empty,
		ztick=\empty,
		view={50}{20},
	},
	axes 3D no lines/.style={
		axis lines=center,
		axis line style={draw=none},
		tick style={draw=none},
		xlabel={},
		ylabel={},
		zlabel={},
		z buffer=sort,
		xmin=-4, xmax=4,
		ymin=-4, ymax=4,
		zmin=-4, zmax=4,
		xtick=\empty,
		ytick=\empty,
		ztick=\empty,
		view={50}{40},
	},
	sequence plot/.style n args={4}{
		vector plane,
		width=10cm, height=7cm,
		xmin=0, xmax=#1,
		ymin=#2, ymax=#3,
		axis line style={-stealth, thick},
		xlabel={$n$},
		ylabel={$#4$},
		xtick={1,...,#1},
		ticklabel style={font=\small},
		domain={0:#1},
		samples=#1+1,
	},
  zoomin/.style n args={2}{
    width=6.5cm, height=6.5cm,
    axis line style={-, thick},
    xlabel={}, ylabel={},
    domain={{#1-#2}:{#1+#2}},
    samples=100,
    xmin={#1-#2}, xmax={#1+#2},
    ymin={#1-#2}, ymax={#1+#2},
    xticklabels={,},
    yticklabels={,},
    grid=both,
    major grid style={draw=gray!5},
    minor grid style={draw=gray!5},
    minor tick num=2,
},
}

\pgfkeys{
	/pgfplots/vector plane/.style={
		axis x line=middle,
		axis y line=middle,
		xlabel=$x$,
		ylabel=$y$,
		every axis x label/.style={
			at={(ticklabel* cs:1.02)},
			anchor=west,
		},
		every axis y label/.style={
			at={(ticklabel* cs:1.02)},
			anchor=south,
		},
		axis line style={stealth-stealth, thick},
		label style={font=\large},
		tick label style={font=\large},
		samples=100,
		xmin=-5, xmax=5,
		ymin=-5, ymax=5,
		domain=-5:5,
		grid=both,
		major grid style={black!5},
		minor grid style={black!5},
	},
	/pgfplots/tapir frame/.style={
		anchor=center,
		hide axis,
		axis lines=left,
		xtick=\empty,
		ytick=\empty,
		xmin=-0.6, xmax=0.6,
		ymin=-0.5, ymax=0.5,
	},
}

\pgfplotscreateplotcyclelist{xcolorscl}{%
  {xred, mark=*, thick},
  {xblue, mark=*, thick},
  {xgreen, mark=*, thick},
  {xpurple, mark=*, thick},
  {xorange, mark=*, thick},
  {xcyan, mark=*, thick},
  {xpink, mark=*, thick},
}

\newcommand{\tapirTransComp}[7]{
	\pgftransformcm{#1}{#2}{#3}{#4}{\pgfpoint{#5}{#6}}
	\begin{axis}[tapir frame]
		\node (#7) at (0,0) {\includesvg[scale=0.75]{figures/linear_algebra/tapir_transform2}};
	\end{axis}
	\pgftransformreset
}			

%-------%
% FONTS %
%-------%

\usepackage{fontawesome}
\usepackage{kpfonts}

%--------%
% TABLES %
%--------%

\usepackage{longtable}
\usepackage{multirow}
\usepackage{array}
\def\tabcol{xpurple}
\captionsetup[table]{labelfont={color=\tabcol,bf}}

%--------%
% COLORS %
%--------%

\usepackage{xcolor}

% Normal colors
\definecolor{xred}{HTML}{BD4242}
\definecolor{xblue}{HTML}{4268BD}
\definecolor{xgreen}{HTML}{52B256}
\definecolor{xpurple}{HTML}{7F52B2}
\definecolor{xorange}{HTML}{FF6E17}
\definecolor{xdotted}{HTML}{999999}
\definecolor{xgray}{HTML}{777777}
\definecolor{xcyan}{HTML}{80F5DC}
\definecolor{xpink}{HTML}{F690EA}
\definecolor{xgrayblue}{HTML}{49B095}
\definecolor{xgraycyan}{HTML}{5AA1B9}
\definecolor{xturquoise}{HTML}{37D190}

% Dark colors
\colorlet{xdarkred}{red!85!black}
\colorlet{xdarkblue}{xblue!85!black}
\colorlet{xdarkgreen}{xgreen!85!black}
\colorlet{xdarkpurple}{xpurple!85!black}
\colorlet{xdarkorange}{xorange!85!black}
\colorlet{xdarkcyan}{xcyan!85!black}

% Very dark colors
\colorlet{xverydarkblue}{xblue!50!black}

% Document-specific colors
\colorlet{normaltextcolor}{black}
\colorlet{figtextcolor}{xblue}

% Enumerated colors
\colorlet{xcol0}{black}
\colorlet{xcol1}{xred}
\colorlet{xcol2}{xblue}
\colorlet{xcol3}{xgreen}
\colorlet{xcol4}{xpurple}
\colorlet{xcol5}{xorange}
\colorlet{xcol6}{xcyan}
\colorlet{xcol7}{xpink!75!black}

% Blue-Purple (should just used colorbrewer...)
\definecolor{xrainbow0}{HTML}{e41a1c}
\definecolor{xrainbow1}{HTML}{a24057}
\definecolor{xrainbow2}{HTML}{606692}
\definecolor{xrainbow3}{HTML}{3a85a8}
\definecolor{xrainbow4}{HTML}{42977e}
\definecolor{xrainbow5}{HTML}{4aaa54}
\definecolor{xrainbow6}{HTML}{629363}
\definecolor{xrainbow7}{HTML}{7e6e85}
\definecolor{xrainbow8}{HTML}{9c509b}
\definecolor{xrainbow9}{HTML}{c4625d}
\definecolor{xrainbow10}{HTML}{eb751f}
\definecolor{xrainbow11}{HTML}{ff9709}

%------- %
% XHFILL %
%------- %

\usepackage{xhfill}

%---------------%
% CHAPTER TITLE %
%---------------%

% Used to set chapter title look.
% "explicit" is there so we can
% use the first parameter (#1).
\usepackage[explicit]{titlesec}
\newcommand{\chapternumsize}{\fontsize{70}{70}\selectfont}
\newcommand{\chaptertitlesize}{\fontsize{30}{30}\selectfont}

\titleformat{\chapter}{\sffamily\bfseries}{}{0pt}{
	\centering
	\begin{tikzpicture}[font=\sffamily\bfseries]
		\fill[black!10, rounded corners] (0,0) rectangle (5cm,5cm);
		\node (chapternum) at (2.5cm,4.5cm) {\fontsize{85}{85}\selectfont\thechapter};
		\node[below of=chapternum, yshift=-2.5cm] {\includesvg[scale=1.5]{figures/chapters/\pgfkeysvalueof{/xref dict/\thechapter}}};
		\node[align=center, below of=chapternum, yshift=-6cm] {\fontsize{30}{30}\selectfont\uppercase{#1}};
		\node[align=center, above of=chapternum, yshift=5mm] {\huge CHAPTER};
	\end{tikzpicture}
}

\titleformat{\section}
  {\normalfont\Large\bfseries}% format
  {}% label
  {0pt}% sep
	{\titlerule\newline\thetitle~~~\uppercase{#1}}% before code
  [{\titlerule[0.4pt]}]% after code

%--------%
% TABLES %
%--------%

\usepackage{array}
\newcolumntype{+}{>{\global\let\currentrowstyle\relax}}
\newcolumntype{^}{>{\currentrowstyle}}
\newcommand{\rowstyle}[1]{\gdef\currentrowstyle{#1}%
  #1\ignorespaces
}

%------------%
% REFERENCES %
%------------%

\usepackage{chngcntr}
\counterwithin{table}{chapter}
\counterwithin{figure}{chapter}
\def\chapterautorefname{Chapter}
\def\sectionautorefname{Section}
\makeatletter
\def\tcb@cnt@exampleautorefname{Example}
\def\tcb@cnt@noteautorefname{Note}
\makeatother

%---------%
% HEADERS %
%---------%

\usepackage{fancyhdr}
\pagestyle{fancy}
\fancyhf{}% Clear header/footer
\renewcommand{\chaptermark}[1]{\markboth{\chaptername\ \thechapter:\ #1}{}}
\fancyhead[RO,LE]{\leftmark}% Chapter details in book
\fancyfoot[RO,LE]{\thepage}
\makeatother
\def\MakeNoNewlines#1{\begingroup\def\\{ }#1\endgroup}
\renewcommand{\chaptermark}[1]{\markboth{Chapter \thechapter:\ \MakeNoNewlines{#1}}{}}

%-------%
% BOXES %
%-------%

\usepackage[most]{tcolorbox}

\tikzset{
	second box/.style={
		anchor=east,
		text=white,
		rounded corners,
		fill=#1,
		xshift=-4mm,
	},
}

\tcbset{
	common/.style n args={2}{
		colframe={#1},
		colback={#1!5},
		colbacktitle={#1},
		lower separated=false,
		coltitle=white,
		boxrule=1pt,
		fonttitle=\bfseries,
		enhanced,
		breakable,
		top=8pt,
		before skip=1em,
		after skip=2em,
		attach boxed title to top left={
			yshift=-0.25cm,
			xshift=0.38cm,
		},
		boxed title style={
			boxrule=0pt,
			colframe=white,
			arc=5pt,
			outer arc=4pt,
		},
		separator sign={~~},
		overlay unbroken and last={
			\node[text=white, align=right, rounded corners, fill=#1, xshift=-4mm, minimum height=6mm, anchor=east] at (frame.south east) {#2};
		}
	},
	defstyle/.style={
		common={xpurple}{$\bm{\pi}$},
	},
	theoremstyle/.style={
		common={xgraycyan}{$\multimapdotbothA$},
	},
	lemmastyle/.style={
		common={xgrayblue}{$\multimap$},
	},
	proofstyle/.style={
		common={xgreen}{\textbf{QED}},
	},
	examplestyle/.style={
		common={xblue}{\faStar},
	},
	notestyle/.style={
		common={xred}{\textbf{!}},
	},
	challengestyle/.style={
		common={xorange}{\textbf{?}},
	},
	quotestyle/.style={
		common={gray}{\textbf{"}},
	},
	Summarystyle/.style={
		common={xturquoise}{\textbf{\faPuzzlePiece}},
	},
}

\newtcbtheorem[auto counter, number within=chapter]{definition}{Definition}{defstyle}{def}
\newtcbtheorem[auto counter, number within=chapter]{theorem}{Theorem}{theoremstyle}{theorem}
\newtcbtheorem[auto counter, number within=chapter]{lemma}{Lemma}{lemmastyle}{lemma}
\newtcbtheorem[auto counter, number within=chapter]{proof}{Proof}{proofstyle}{proof}
\newtcbtheorem[auto counter, number within=chapter]{example}{Example}{examplestyle}{example}
\newtcbtheorem[auto counter, number within=chapter]{note}{Note}{notestyle}{note}
\newtcbtheorem[auto counter, number within=chapter]{challenge}{Challenge}{challengestyle}{challenge}
\newtcbtheorem[auto counter, number within=chapter]{summary}{Summary}{Summarystyle}{summary}
\newtcbtheorem[]{nquote}{Quote}{quotestyle}{quote}

%---------%
% FIGURES %
%---------%

\usepackage{float}
\usepackage{caption}
\usepackage[colorlinks=true, linkcolor=xblue]{hyperref}
\usepackage{subcaption}
\usepackage{cleveref}
\captionsetup[subfigure]{subrefformat=simple,labelformat=simple}
\renewcommand\thesubfigure{(\alph{subfigure})}

% Caption label
\DeclareCaptionLabelSeparator{doublespace}{\ \ }
\captionsetup{labelfont={color=xblue,bf}, figurename=Figure, labelsep=doublespace}

% Refs format
\newcommand{\xref}[2][fig]{
	\color{figtextcolor}\textbf{\pgfkeysvalueof{/xref dict/#1}}~\ref{#1:#2} \color{normaltextcolor}
}

%-------------------%
% HIGHLIGHTED WORDS %
%-------------------%

\usepackage{imakeidx}
\usepackage{marginnote}
\makeindex
\renewcommand\emph[1]{\color{xpurple}{\textbf{#1}}\color{normaltextcolor}\index{#1}}%\marginnote[#1]{}

%-------%
% MATHS %
%-------%

\usepackage{amsmath, bm, mathtools, commath, relsize}
\usepackage{physics}
\allowdisplaybreaks
\usepackage{nicematrix}
\numberwithin{equation}{section}
\usepackage{siunitx}
\usepackage[thicklines]{cancel}
\renewcommand\CancelColor{\color{xred}}
\newcommand{\cancelcol}[2][xred]{ % This is such a silly solution...
	\renewcommand\CancelColor{\color{#1}}
	\cancel{#2}
	\renewcommand\CancelColor{\color{xred}}
}
\newcommand{\Rs}[1][]{\mathbb{R}^{#1}}
\newcommand{\Cs}[1][]{\mathbb{C}^{#1}}
\newcommand{\stcomp}[1]{{#1}^\complement}
%\newcommand{\true}{\colorbox{xblue!25}{true}}
%\newcommand{\false}{\colorbox{xred!25}{false}}
\newcommand{\true}{\textcolor{xblue}{\textbf{true}}}
\newcommand{\false}{\textcolor{xred}{\textbf{false}}}
\newcommand{\AND}{\textbf{AND}}
\newcommand{\OR}{\textbf{OR}}
\newcommand{\opand}{\wedge}
\newcommand{\opor}{\vee}
\newcommand{\falseprop}[1]{
	\begingroup
	\color{xred}
	\underset{\false}{#1}
	\endgroup
}
\newcommand{\trueprop}[1]{
	\begingroup
	\color{xblue}
	\underset{\true}{#1}
	\endgroup
}
\newcommand{\defeq}{:=}
\newcommand{\eqdef}{=:}
\newcommand{\conj}[1]{\overline{#1}}
\newcommand{\sconj}[1]{#1^{*}}
\newcommand{\nroot}[2]{
	\sqrt[\leftroot{3}\uproot{3}#1]{#2}
}
\newcommand{\downmapsto}{\rotatebox[origin=c]{-90}{$\scriptstyle\mapsto$}\mkern2mu}
\newcommand{\inpr}[2]{\langle #1,#2 \rangle}
\newcommand{\iu}{\mathrm{i}\mkern1mu}
\newcommand{\eu}{\mathrm{e}}
\newcommand{\Eu}[1]{\mathrm{e}^{#1}}
\newcommand{\EU}[1]{\exp\left(#1\right)}
\newcommand{\Z}[1]{\mathbb{Z}_{#1}}
\renewcommand{\gcd}[2]{\text{gcd}\left(#1,#2\right)}
\newcommand{\Ctrig}{\text{c}}
\newcommand{\Strig}{\text{s}}
\newcommand{\Refl}{\text{Ref}}
\renewcommand{\tr}{\text{tr}}
\newcommand{\adj}{\text{adj}}
\newcommand{\floor}[1]{\lfloor #1 \rfloor}
\newcommand{\ceil}[1]{\lceil #1 \rceil}
\newcommand{\Err}{\text{Err}}

% Colored math symbols
\newcommand{\xr}[1]{\color{xred}{#1}\color{normaltextcolor}}
\newcommand{\xb}[1]{\color{xblue}{#1}\color{normaltextcolor}}
\newcommand{\xg}[1]{\color{xgreen}{#1}\color{normaltextcolor}}
\newcommand{\xdg}[1]{\color{xdarkgreen}{#1}\color{normaltextcolor}}
\newcommand{\xp}[1]{\color{xpurple}{#1}\color{normaltextcolor}}
\newcommand{\xo}[1]{\color{xorange}{#1}\color{normaltextcolor}}
\newcommand{\xc}[1]{\color{xcyan!75!black}{#1}\color{normaltextcolor}}
\newcommand{\xpk}[1]{\color{xpink!85!black}{#1}\color{normaltextcolor}}

% Highlights
\newcommand{\rhl}[1]{\textcolor{xred}{#1}}
\newcommand{\bhl}[1]{\textcolor{xblue}{#1}}
\newcommand{\ghl}[1]{\textcolor{xgreen}{#1}}
\newcommand{\phl}[1]{\textcolor{xpurple}{#1}}

% Row and Column vectors
\makeatletter
\newcommand\rcvector[2][\\]{\ensuremath{%
  \global\def\rc@delim{#1}%
    \negthinspace\begin{bmatrix}
      \rc@vector #2;\relax\noexpand\@eolst%
    \end{bmatrix}}}
\def\rc@vector #1;#2\@eolst{%
  \ifx\relax#2\relax
    #1
  \else
    #1\rc@delim
    \rc@vector #2\@eolst%
  \fi}
\makeatother

\newcommand{\colvec}{\rcvector}
\newcommand{\rowvec}[1]{\rcvector[,\;]{#1}}

% Colored 2D and 3D column vectors
\newcommand{\xtwocolvec}[2]{
  \colvec{\xr{#1};\xb{#2}}
}
\newcommand{\xthreecolvec}[3]{
  \colvec{\xr{#1};\xb{#2};\xdg{#3}}
}

% Colored (common) vectors and angles I use
\renewcommand{\vu}{\xr{\vec{u}}}
\newcommand{\vv}{\xb{\vec{v}}}
\newcommand{\vw}{\xp{\vec{w}}}
\newcommand{\ath}{\textcolor{xpurple}{\theta}}
\newcommand{\vutd}{\colvec{\textcolor{xred}{a};\textcolor{xred}{b}}}
\newcommand{\vvtd}{\colvec{\textcolor{xblue}{c};\textcolor{xblue}{d}}}
\newcommand{\rcu}[1]{\xr{u^{#1}}}
\newcommand{\bcv}[1]{\xb{v^{#1}}}

% Norm of a whatever
\newcommand{\gnorm}[1]{\left\|\,  #1 \, \right\|}

% Norm of a vector
\renewcommand{\norm}[1]{\left\|\,  \vec{#1} \, \right\|}

% Projection
\newcommandx*{\projection}[4][1=u, 2=v, 3=xred, 4=xblue]{
  \text{proj}_{\textcolor{#4}{\vec{#2}}}\textcolor{#3}{\vec{#1}}
}
\newcommand{\proj}[2]{
	\text{proj}_{#1}{#2}
}

% Standard basis vectors
\newcommand{\eb}[1]{\hat{e}_{#1}}
\newcommand{\ebv}[1]{\vec{e}_{#1}}

% Colored x,y,z hats:
\newcommand{\cxhat}{\xr{\hat{x}}}
\newcommand{\cyhat}{\xb{\hat{y}}}
\newcommand{\czhat}{\xdg{\hat{z}}}

% Normal vector
\newcommand{\normalVec}[2]{
	\hat{#1}_{\mathbf{#2}}
}

% Color-coded matrix elements
\newcommand{\Matel}[5]{
	#1_{\textcolor{#2}{#4}\textcolor{#3}{#5}}
}
\newcommand{\Ma}[2]{
	\Matel{a}{xred}{xblue}{#1}{#2}
}
\newcommand{\Mb}[2]{
	\Matel{b}{xred}{xblue}{#1}{#2}
}
\newcommand{\Mm}[2]{
	\Matel{m}{xred}{xblue}{#1}{#2}
}
\newcommand{\Ui}[1]{
	u_{\textcolor{xred}{#1}}
}
\newcommand{\Uj}[1]{
	u_{\textcolor{xblue}{#1}}
}
\newcommand{\Vi}[1]{
	v_{\textcolor{xred}{#1}}
}
\newcommand{\Vj}[1]{
	v_{\textcolor{xblue}{#1}}
}

% Get the integer and fraction parts of a decimal number
\makeatletter
\newcommand{\intpart}[1]{\expandafter\int@part#1..\@nil}
\def\int@part#1.#2.#3\@nil{\if\relax#1\relax0\else#1\fi}
\newcommand{\fracpart}[1]{\expandafter\frac@part#1..\@nil}
\def\frac@part#1.#2.#3\@nil{\if\relax#2\relax0\else#2\fi}
\makeatother

% Non-linear transformations
\makeatletter
\def\nltransA{%
	\pgfmathsetmacro{\myX}{\pgf@x+0.0025*\pgf@y^2+0.3*\pgf@x+3}
	\pgfmathsetmacro{\myY}{\pgf@y+0.0050*\pgf@x^2-5}
	\setlength{\pgf@x}{\myX pt}
	\setlength{\pgf@y}{\myY pt}
}
\def\nltransB{%
	\pgfmathsetmacro{\myX}{\pgf@x}
	\pgfmathsetmacro{\myY}{\pgf@y+0.1*abs(\pgf@x)^1.01}
	\setlength{\pgf@x}{\myX pt}
	\setlength{\pgf@y}{\myY pt}
}
\makeatother

% 2D figure for linear transformations
\newcommand{\tapirTrans}[5]{
	\begin{tikzpicture}
		\pgftransformcm{#1}{#2}{#3}{#4}{\pgfpoint{0}{0}}
		\begin{axis}[
			vector plane,
			anchor=center,
			width=#5, height=#5,
			xmin=-1, xmax=1,
			ymin=-1, ymax=1,
			xlabel={},
			ylabel={},
			xtick={-1,-0.9,...,1},
			ytick={-1,-0.9,...,1},
			xticklabels={,,},
			yticklabels={,,},
		]
		\pgfmathsetmacro{\sc}{#5/8cm}
		\node at (0,0) {\includesvg[scale=\sc]{figures/linear_algebra/tapir_transform2}};
	\end{axis}
	\pgftransformreset
	\end{tikzpicture}
}
\newcommand{\tapirTransCM}[7]{
	\pgftransformcm{#1}{#2}{#3}{#4}{\pgfpoint{#5}{#6}}
	\pgfmathsetmacro{\sc}{#7/8cm}
	\node at (0,0) {\includesvg[scale=\sc]{figures/linear_algebra/tapir_transform2}};
	\pgftransformreset
}

\newcommand{\IdMatHl}[3]{
	\node[fill=#3, fit=(#1-#2-#2)(#1-#2-#2)] {};
}

\newcommand{\veccomp}[4]{
	\draw[thin, dashed, #4!50] (0,#3) -- (#2,#3) -- (#2,0);
	\draw[vector, #4] (0,0) -- (#2,#3);
	\node[text=#4] at ({1.2*#2},{1.2*#3}) {$\vec{#1}=\colvec{#2;#3}$};
}

\newcommand{\Null}{\text{Null}}
\renewcommand{\rank}{\text{rank}}

%-------------%
% OTHER STUFF %
%-------------%

\usepackage[autostyle]{csquotes} % for quotes
\usepackage[]{quoting}

\usepackage[version=4]{mhchem} % for chemistry
\newcommand{\shrug}[1][]{% shrug emoji
	\begin{tikzpicture}[baseline,x=0.8\ht\strutbox,y=0.8\ht\strutbox,line width=0.125ex,#1]
		\def\arm{(-2.5,0.95) to (-2,0.95) (-1.9,1) to (-1.5,0) (-1.35,0) to (-0.8,0)};
		\draw \arm;
		\draw[xscale=-1] \arm;
		\def\headpart{(0.6,0) arc[start angle=-40, end angle=40,x radius=0.6,y radius=0.8]};
		\draw \headpart;
		\draw[xscale=-1] \headpart;
		\def\eye{(-0.075,0.15) .. controls (0.02,0) .. (0.075,-0.15)};
		\draw[shift={(-0.3,0.8)}] \eye;
		\draw[shift={(0,0.85)}] \eye;
% draw mouth
		\draw (-0.1,0.2) to [out=15,in=-100] (0.4,0.95);
\end{tikzpicture}
}

% List enumeration with chapter number
\usepackage{enumitem}
% \setenumerate[1]{label=\thechapter.\arabic*.}
% \setenumerate[2]{label*=\arabic*.}

% Lists that have bold/underline labels

\newcommand{\descitemlabel}[1]{%
  \textbullet\ \textbf{#1}:%
}

\newcommand{\listitemlabel}[1]{%
	% Instead of bullet there should be a counter here,
	% but it doesn't work :(
	\textbullet\ \underline{#1}:%
}

\newenvironment{descitemize}
{\begin{description}[itemsep=1.5em, before=\let\makelabel\descitemlabel]}
{\end{description}}

\newenvironment{listitemize}
{\begin{description}[leftmargin=0pt, itemsep=1.25em, before=\let\makelabel\listitemlabel]}
{\end{description}}

% Ornaments
\usepackage{pgfornament}

% Colored matrix cells
\newcommand{\ceco}[1]{\cellcolor{#1}}
\newcommand{\cxred}{\ceco{xred!20}}
\newcommand{\cxblue}{\ceco{xblue!20}}
\newcommand{\cxgreen}{\ceco{xgreen!20}}
\newcommand{\cxpurple}{\ceco{xpurple!20}}
\newcommand{\cxorange}{\ceco{xorange!20}}
\newcommand{\cxcyan}{\ceco{xcyan!20}}
\newcommand{\cxpink}{\ceco{xpink!20}}

% Define identity matrix using NiceMatrix 
\NewDocumentCommand{\Identity}{O{}m}
{\pAutoNiceMatrix[#1]{#2-#2}{\ifthenelse{\arabic{iRow}=\arabic{jCol}}{1}{0}}}

% Define highlighted identity matrix
\newcommand{\IdentityHl}[3]{
	$\Identity[name=#2]{#1}$
	\begin{tikzpicture}[overlay, remember picture, blend mode=multiply]
		\foreach \k in {1,...,#1}{
			\node[mathl={#3}] at (#2-\k-\k) {}; 
		}
	\end{tikzpicture}
}

% Matrix highlight from list
\newcommand{\MatHL}[2]{
	\foreach \el in {#1}{
		\node[mathl={#2}] at \el {};
	}
}

% Minor
\newcommand{\MatMinor}[4]{
	\foreach \d in {1,...,#2}{
		\node[mathl={black!50}] at (#1-#3-\d) {};
		\node[mathl={black!50}] at (#1-\d-#4) {};
	}
}

% Matrix-matrix product illustration
\newcommand{\boxgrid}[6]{
  \pgfmathsetmacro{\a}{#1}
  \pgfmathsetmacro{\nrow}{#2}
  \pgfmathsetmacro{\ncol}{#3}
  \pgfmathsetmacro{\dx}{#4}
  \pgfmathsetmacro{\dy}{#5}
  \foreach \row in {1,...,\nrow}{
  \foreach \col in {1,...,\ncol}{
  \draw[thick] (\row*\a+\dx,-\col*\a-\dy) rectangle ++(\a,-\a) node [midway] {$#6_{\col\row}$};
  }
  }
  \draw[thick] (\dx+\a*1.7,\dy-\a*0.7) -- ++(-\a*2,0);
}

% Augmented (nice)matrix
\makeatletter
\ExplSyntaxOn
\NewDocumentEnvironment { bNiceMatrixAug } { }
  { \begin { bNiceMatrix } }
  { 
    \CodeAfter \tikz \draw (1-|\arabic{jCol}) -- (last-|\arabic{jCol}) ; 
    \end { bNiceMatrix } 
  }

\ExplSyntaxOff
\makeatother

% Test: solve symbol
\newcommand{\solvesym}[1]{
	\tikz[overlay, remember picture]{
		\node[right of=#1] {\includesvg[scale=0.05]{figures/symbols/solve_hand_only}};
	}
}

\newcommand{\tbw}[1]{
  \begin{tcolorbox}[colback=xred!10!white, colframe=xred!30, coltitle=black, title=\faExclamationCircle\ To write/do \faExclamationCircle]
    #1
  \end{tcolorbox}
}

% Testing boldface in titles
% \usepackage{etoolbox}
% \makeatletter
% % \tracingpatches
% \patchcmd{\@sect}{#8}{\boldmath #8}{}{}
% \let\ori@chapter\@chapter
% \def\@chapter[#1]#2{\ori@chapter[\boldmath#1]{\boldmath#2}}
% \makeatother

% Testing sums in PGFPlots
\newcounter{isum}
\pgfplotsset{summand/.initial=max}
\pgfmathdeclarefunction{sum}{2}{%
\begingroup%
\pgfkeys{/pgf/fpu,/pgf/fpu/output format=fixed}%
\edef\myfun{\pgfkeysvalueof{/pgfplots/summand}}%
\pgfmathsetmacro{\mysum}{0}%
\pgfmathsetmacro{\myx}{#2}%
\pgfmathtruncatemacro{\imax}{#1}%
\setcounter{isum}{1}%
\loop
\pgfmathsetmacro{\mysum}{\mysum+\myfun(\value{isum},#2)}%
\ifnum\value{isum}<\imax\relax
\stepcounter{isum}\repeat
\pgfmathparse{\mysum}%
\pgfmathsmuggle\pgfmathresult\endgroup%
}%

\newcommand{\SignOf}[1]{
  \ifthenelse{#1<0}{<}{
    \ifthenelse{#1>0}{>}{=}
  }
}
