\section{Vectors}
\subsection{Basics}
\emph{Vectors} are the fundamental objects of linear algebra: the entire field revolves around manipulation of vectors. In this chapter we deal with the so-called \emph{real vectors}, which can be be defined in a geometric way:

\begin{definition}{Real vectors}{real vectors}
	A \textit{real vector} is an object with a \emph{magnitude} (also called \emph{norm}) and a \emph{direction}.
\end{definition}

In this chapter we refer to real vectors simply as \textit{vectors}. Vectors can have $1,2,3,\dots$ number of dimensions. $2$-dimensional vectors can be drawn as arrows on a plane:
\begin{center}
  \begin{tikzpicture}
    \draw[vector, xred] (0,0) -- ++(2,3);
    \draw[vector, xblue] (-1,0) -- ++(-1,2);
    \draw[vector, xgreen] (0,-1) -- ++(-3,0);
    \draw[vector, xpurple] (2,0) -- ++(-1,-3);
    \draw[vector, xorange] (-4,2) -- ++(0,-4);
    \draw[vector, black] (-7,1) -- ++(1,-1);
  \end{tikzpicture}
\end{center}

Similarly, $3$-dimensional vectors can be drawn as arrows in space:
% Work needs to be done on this illustration.
\begin{center}
	\centering
	\tdplotsetmaincoords{70}{45}
	\begin{tikzpicture}[tdplot_main_coords]
    \draw[vector, xorange] (0,0,0) -- (2,-3,-3);
		\draw[black!40, very thick, opacity=1, fill=gray, fill opacity=0.2] (4,4,0) -- (4,-4,0) -- (-4,-4,0) -- (-4,4,0) -- cycle;
    \draw[vector, xred] (0,0,0) -- (3,2,2);
    \draw[vector, xblue] (-2,-1,0) -- (-3,-1,3);
    \draw[vector, xpurple] (-3,3,0) -- ++(3,-4,0.5);
	\end{tikzpicture}
\end{center}

Unfortunately, it is difficult (if not impossible) to draw higher-dimensional vectors. For now, we will concentrate on $2$-dimensional vectors and explore their properties. Later in the section we will apply what we learned from $2$-dimensional vectors to $3$-dimensional vectors and higher-dimensional vectors as well.

Vectors are usually denoted in one of the following ways:

\begin{descitemize}
	\setlength\itemsep{1em}
	\addtolength{\itemindent}{5mm}
	\item[Arrow above letter] $\vec{u},\ \vec{v},\ \vec{x},\ \vec{a},\ \dots$
	\item[Bold letter] $\bm{u},\ \bm{v},\ \bm{x},\ \bm{a},\ \dots$
	\item[Bar below letter] $\underline{u},\ \underline{v},\ \underline{x},\ \underline{a},\ \dots$
\end{descitemize}

In this book we use the first notation style, i.e. an arrow above the letter. In addition vectors will almost always be denoted using lowercase \textit{Latin} script.

When discussing vectors in a single context, we always consider them starting at the same point, called the \emph{origin}, and \emph{translating} (moving) vectors around in space does not change their properties: only their norms and directions matter. For example, we can draw the $2$-dimensional vectors from before such that their origins all lie on the same point:

\begin{center}
	\begin{tikzpicture}
		\draw[vector, xred] (0,0) -- ++(2,3);
		\draw[vector, xblue] (0,0) -- ++(-1,2);
		\draw[vector, xgreen] (0,0) -- ++(-3,0);
		\draw[vector, xpurple] (0,0) -- ++(-1,-3);
		\draw[vector, xorange] (0,0) -- ++(0,-4);
		\draw[vector, black] (0,0) -- ++(1,-1);
		\fill (0,0) circle (0.05);
	\end{tikzpicture}
\end{center}

Any vector can be scaled by a real number $\alpha$: when this happens, its norm is multiplied by $\alpha$ while its direction stays the same. We call $\alpha$ a \emph{scalar}. For example, the vector $\color{xred}{\vec{v}}$ on the left is scaled here by different scalars $\alpha=2,2.5,-1$ and $-2$:

\begin{center}
	\begin{tikzpicture}[every node/.style={midway, left, xshift=-2mm}]
		\Large
		\draw[vector, xred] (0,0) -- ++(1.5,1) node {$\vec{v}$};
		\draw[vector, xblue] (2,0) -- ++(3,2) node {$2\cdot \vec{v}$};
		\draw[vector, xpurple] (4.5,0) -- ++(3.75,2.5) node {$2.5\cdot \vec{v}$};
		\draw[stealth-, thick, xgreen!85!black] (7.5,0) -- ++(1.5,1) node {$-1\cdot \vec{v}$};
		\draw[stealth-, thick, black] (9.5,0) -- ++(3,2) node {$-2\cdot \vec{v}$};
	\end{tikzpicture}
\end{center}

\begin{note}{Negative scale}{negative scale}
	As can be seen in the example above, when scaling a vector by a negative amount its direction reverses. However, we consider two opposing direction (i.e. directions that are $\ang{180}$ apart) as being the same direction.
\end{note}

In this chapter we use the following notation for the norm of a vector $\vec{v}$: $\norm{v}$.

A vector $\vec{v}$ with norm $\norm{v}=1$ is called a \emph{unit vector}, and is usually denoted by replacing the arrow symbol by a hat symbol: $\hat{v}$. Any vector (except $\vec{0}$) can be scaled into a unit vector by scaling  the vector by $1$ over its own norm, i.e.
\begin{equation}
	\hat{v} = \frac{1}{\norm{v}}\vec{v}.
	\label{eq:normalized vector}
\end{equation}
The result of normalization is a vector of unit norm which points in the same direction of the original vector.

Two vectors can be added together to yield a third vector: $\vu+\vv=\vw$. To find $\vw$ we use the following procedure (depicted in \autoref{fig:vector addition geometric}):
% The items need to be typeset without the chapter number
\begin{enumerate}
	\item Move (translate) $\vv$ such that its origin lies on the head of $\vu$.
	\item The vector $\vw$ is the vector drawn from the origin of $\vu$ to the head of $\vv$.
\end{enumerate}

\renewcommand\thesubfigure{\arabic{subfigure}}
\begin{figure}[h]
	\centering
	 \begin{subfigure}[t]{0.45\textwidth}
		\centering
		\begin{tikzpicture}
			\coordinate (O) at (0,0);
			\coordinate (u) at (-2,1);
			\coordinate (v) at (1.5,1);
			\coordinate (w) at ($(u)+(v)$);
			\draw[vector, xred] (O) -- (u) node[above left] {$\vu$};
			\draw[vector, xblue] (O) -- (v) node[above right] {$\vv$};
			\draworigin
		\end{tikzpicture}
		\caption{The vectors $\vu$ and $\vv$.}
	\end{subfigure}
	\hfill
	\begin{subfigure}[t]{0.45\textwidth}
		\centering
		\begin{tikzpicture}
			\draw[vector, xred] (O) -- (u) node[above left] {$\vec{u}$};
			\draw[vector, xblue] (u) -- ++(v) node[above right] {$\vec{v}$};
			\draworigin
		\end{tikzpicture}
		\caption{Translating $\vv$ such that its origin lies at the head of $\vu$.}
	\end{subfigure}

	\vspace{3em}
	\begin{subfigure}[t]{0.45\textwidth}
		\centering
		\begin{tikzpicture}
			\draw[vector, xred] (O) -- (u) node[above left] {$\vec{u}$};
			\draw[vector, xblue] (u) -- ++(v) node[above right] {$\vec{v}$};
			\draw[vector, xpurple] (O) -- (w) node[right, yshift=-2mm] {$\vec{w}$};
			\draworigin
		\end{tikzpicture}
		\caption{Drawing the vector $\vw$ from the origin to the head of $\vv$.}
	\end{subfigure}
	\hfill
	\begin{subfigure}[t]{0.45\textwidth}
		\centering
		\begin{tikzpicture}
			\draw[vector, xred] (O) -- (u) node[above left] {$\vec{u}$};
			\draw[vector, xblue] (O) -- (v) node[above right] {$\vec{v}$};
			\draw[vector, xpurple] (O) -- (w) node[above] {$\vec{w}$};
			\draworigin
		\end{tikzpicture}
		\caption{All three vectors with the same origin.}
	\end{subfigure}
	\caption{Vector addition.}
	\label{fig:vector addition geometric}
\end{figure}

The addition of vectors as depicted here is commutative, i.e. $\vu+\vv = \vv+\vu$. This can be seen by using the \emph{parallelogram law of vector addition} as depicted in \autoref{fig:parallelogram}: drawing the two vectors $\vu, \vv$ and their translated copies (each such that its origin lies on the other vector's head) results in a parallelogram.

\begin{figure}[h]
	\centering
	\begin{tikzpicture}
		\draw[vector, xred] (O) -- (u) node[above left] {$\vec{u}$};
		\draw[vector, xblue] (O) -- (v) node[above right] {$\vec{v}$};
		\draw[vector, xred] (v) -- ++(u);
		\draw[vector, xblue] (u) -- ++(v);
		\draw[vector, xpurple] (O) -- (w) node[above] {$\vec{w}$};
		\draworigin
	\end{tikzpicture}
	\caption{The parallogram law of vector addition.}
	\label{fig:parallelogram}
\end{figure}

An important vector is the \emph{zero-vector}, denoted as $\vec{0}$. The zero-vector has a unique property: it is neutral in respect to vector addition, i.e. for any vector $\vec{v}$,
\begin{equation}
	\vec{v} + \vec{0} = \vec{v}.
	\label{eq:zero-vector}
\end{equation}
(we also say that $\vec{0}$ is the \emph{additive identity} in respect to vectors.)

Any vector $\vec{v}$ always has an \emph{opposite} vector, denoted $-\vec{v}$. The addition of a vector and its opposite always result in the zero-vector, i.e.
\begin{equation}
	\vec{v} + \left( -\vec{v} \right) = \vec{0}.
	\label{eq:opposite vector}
\end{equation}

\subsection{Column representation}
In order to be able to use vectors for actual calculations we must somehow quantify their properties. When quantifying geometric shapes we often first define some unit of measurement, for example a centimeter (\si{cm}). We then use this unit to measure the length of different objects.

While this simple approach works well for describing lengths, in the case of vectors we also want to quantify directions - which becomes a bit complicated in higher dimensions if we use angles. Instead, we use more than one unit of measurement; in fact, we use a vector as a unit of measurement for each dimension (and call these vectors \emph{basis vectors}). For example, we can choose the following two $2$-dimensional vectors $\xr{\vec{e}_{1}}$ and $\xb{\vec{e}_{2}}$ to be used as basis vectors:

\begin{center}
  \begin{tikzpicture}
    \coordinate (O) at (0,0);
    \coordinate (e1) at (1,0.5);
    \coordinate (e2) at (-0.5,0.5);
    \draw[vector, xred] (O) -- ++(e1) node[midway, above] {$\vec{e}_{1}$};
    \draw[vector, xblue] (3,0) -- ++(e2) node[midway, above right] {$\vec{e}_{2}$};
  \end{tikzpicture}
\end{center}

We can then use these basis vectors to measure other $2$-dimensional vectors, for example the following vector $\xp{\vec{v}}$:

\begin{center}
  \begin{tikzpicture}
    \coordinate (v) at ($3*(e1)+2*(e2)$);
    \draw[vector, xpurple] (O) -- ++(v) node[midway, left] {$\vec{v}$};
    \draw[vector, xred, dashed] (O) -- ++(e1) node[midway, below] {$\vec{e}_{1}$};
    \draw[vector, xred, dashed] (e1) -- ++(e1) node[midway, below] {$\vec{e}_{1}$};
    \draw[vector, xred, dashed] ($2*(e1)$) -- ++(e1) node[midway, below] {$\vec{e}_{1}$};
    \draw[vector, xblue, dashed] ($3*(e1)$) -- ++(e2) node[midway, right] {$\vec{e}_{2}$};
    \draw[vector, xblue, dashed] ($3*(e1)+(e2)$) -- ++(e2) node[midway, right] {$\vec{e}_{2}$};
    \fill (O) circle (0.05);
  \end{tikzpicture}
\end{center}

We see that using these basis vectors, $\xp{\vec{v}}=3\xr{\vec{e}_{1}}+2\xb{\vec{e}_{2}}$. This means that we need to add three times $\xr{\vec{e}_{1}}$ and two times $\xb{\vec{e}_{2}}$ to construct $\xp{\vec{v}}$. We denote this fact by writing $\xp{\vec{v}}$ as a column of two numbers:

\[
  \xp{\vec{v}}=\colvec{\tikzmark{v1}{\xr{3}};\tikzmark{v2}{\xb{2}}}
\]

\begin{tikzpicture}[overlay, remember picture]
  \coordinate(v1b) at (pic cs:v1);
  \coordinate(v2b) at (pic cs:v2);
  \node[right of=v1b, anchor=west, yshift=5pt] (v1txt) {How much of $\xr{\vec{e}_{1}}$ is in $\xp{\vec{v}}$};
  \node[right of=v2b, anchor=west, yshift=-5pt] (v2txt) {How much of $\xb{\vec{e}_{2}}$ is in $\xp{\vec{v}}$};
  \draw[->, thick, xred] (v1txt) -- ($(v1b)+(10pt,0)$);
  \draw[->, thick, xblue] (v2txt) -- ($(v2b)+(10pt,0)$);
\end{tikzpicture}

\begin{example}{Vectors from basis vectors}{more vectors same basis}
  Two more vectors represented as sums of the basis vectors $\xr{\vec{e}_{1}}$ and $\xb{\vec{e}_{2}}$:

  \begin{center}
    \begin{tikzpicture}
      \coordinate (u) at ($2*(e1)-2*(e2)$);
      \draw[vector, xorange] (O) -- (u) node[midway, below] {$\xo{\vec{u}}$};
      \draw[vector, xred, dashed] (O) -- ++(e1) node[midway, above] {$\vec{e}_{1}$};
      \draw[vector, xred, dashed] (e1) -- ++(e1) node[midway, above] {$\vec{e}_{1}$};
      \draw[vector, xblue, dashed] ($2*(e1)$) -- ++($-1*(e2)$) node[midway, right] {$-\vec{e}_{2}$};
      \draw[vector, xblue, dashed] ($2*(e1)-(e2)$) -- ++($-1*(e2)$) node[midway, right] {$-\vec{e}_{2}$};
      \fill (O) circle (0.05);
      \node[right of=O, xshift=3cm, anchor=west] (u_math) {$\xo{\vec{u}}=2\xr{\vec{e}_{1}}-2\xb{\vec{e}_{2}}\quad\longrightarrow\quad\xo{\vec{u}}=\xtwocolvec{2}{-2}$};
    \end{tikzpicture}

    \vspace{1.5em}
    \begin{tikzpicture}
      \coordinate (w) at ($-3*(e1)+1.5*(e2)$);
      \draw[vector, xgreen] (O) -- (w) node[midway, above] {$\xg{\vec{w}}$};
      \draw[vector, xred, dashed] (O) -- ++($-1*(e1)$) node[midway, below] {$-\vec{e}_{1}$};
      \draw[vector, xred, dashed] ($-1*(e1)$) -- ++($-1*(e1)$) node[midway, below] {$-\vec{e}_{1}$};
      \draw[vector, xred, dashed] ($-2*(e1)$) -- ++($-1*(e1)$) node[midway, below] {$-\vec{e}_{1}$};
      \draw[vector, xblue, dashed] ($-3*(e1)$) -- ++(e2) node[midway, left] {$\vec{e}_{2}$};
      \draw[vector, xblue, dashed] ($-3*(e1)+(e2)$) -- ++($0.5*(e2)$) node[midway, left] {$\frac{1}{2}\vec{e}_{2}$};
      \fill (O) circle (0.05);
      \node[right of=O, anchor=west] (u_math2) {$\xg{\vec{w}}=-3\xr{\vec{e}_{1}}+1.5\xb{\vec{e}_{2}}\quad\longrightarrow\quad\xg{\vec{w}}=\xtwocolvec{-3}{1.5}$};
    \end{tikzpicture}
  \end{center}
\end{example}

This notation is frequently refered to as a \emph{column vector}, and the numbers are called \emph{coordinates} or \emph{components}. The components themselves are usually denoted using the same symbol used for the vector (without the arrow sign) and an index: for example, the two components of the vector $\xp{\vec{v}}=\xtwocolvec{3}{2}$ are $\xr{v^{1}}=\xr{3}$ and $\xb{v^{2}}=\xb{2}$.

\begin{note}{Index notation}{}
  In this book we use upper-index notation to denote components of column vectors. Do not mistake these for powers! The reason for this choice (as opposed to lowe-index notation, i.e. $v_{1},\ v_{2},$ etc.) is to stay consistent with later parts of the book, where covectors and in general tensors are presented.
\end{note}

The set of basis vectors used to represent vectors as columns is sometimes called a \emph{coordinate system}. We will see different common coordinate systems soon. Vectors have different components in different coordinate systems. For example, we can use the following two vectors $\xdg{\vec{d}_{1}}$ and $\xo{\vec{d}_{2}}$ as basis vectors:

\begin{center}
  \begin{tikzpicture}
    \coordinate (O) at (0,0);
    \coordinate (e1b) at (1,0);
    \coordinate (e2b) at (0,1);
    \draw[vector, xdarkgreen] (O) -- ++(e1b) node[midway, above] {$\vec{d}_{1}$};
    \draw[vector, xorange] (3,0) -- ++(e2b) node[midway, right] {$\vec{d}_{2}$};
  \end{tikzpicture}
\end{center}

In this new coordinate system, the vector $\xp{\vec{v}}$ from earlier has the following column representation:

\vspace{-1.2em}
\begin{center}
  \begin{tikzpicture}
    \coordinate (O) at (0,0);
    \coordinate (vb) at ($2*(e1b)+2.5*(e2b)$);
    \draw[vector, xdarkgreen] (O) -- ++(e1b) node[midway, below] {$\vec{d}_{1}$};
    \draw[vector, xdarkgreen] (e1b) -- ++(e1b) node[midway, below] {$\vec{d}_{1}$};
    \draw[vector, xorange] ($2*(e1b)$) -- ++(e2b) node[midway, right] {$\vec{d}_{2}$};
    \draw[vector, xorange] ($2*(e1b)+(e2b)$) -- ++(e2b) node[midway, right] {$\vec{d}_{2}$};
    \draw[vector, xorange] ($2*(e1b)+2*(e2b)$) -- ++($0.5*(e2b)$) node[midway, right] {$\frac{1}{2}\vec{d}_{2}$};
    \draw[vector, xpurple] (O) -- (vb) node[above] {$\vec{v}$};
    \node[anchor=west] (txt1) at (3.5,1.3) {$\xp{\vec{v}}=2\xdg{\vec{d}_{1}}+2.5\xo{\vec{d}_{2}}\quad\longrightarrow\quad\xp{\vec{v}}=\colvec{\xdg{2};\xo{2.5}}$};
  \end{tikzpicture}
\end{center}
(i.e. its components are $\xdg{v^{1}}=\xdg{2}$ and $\xo{v^{2}}=\xo{2.5}$)

This brings us to an important idea, which unfortunately might confuse those who are new to the topic: \textbf{vectors and their column representation are two separate things!} Vectors are objects with a length (norm) and a direction. They don't ``care'' about how \textit{we} describe them numerically: no matter what coordinate system we use, vectors remain the same - it's their \textit{representation} which changes.

In fact, not only does the choice of coordinate system affect the column representation of all vectors\footnote{except the zero vector, which is always $\vec{0}=\colvec{0;0}$}, two different vectors \textbf{can have the same column representation using two different coordinate systems.}. For example. let's say we choose any two non-zero vectors to be used as basis vectors: $\vec{u}$ and $\vec{v}$. Then the following is always true:
\begin{align*}
  \vec{u} &= 1\cdot\vec{u} + 0\cdot\vec{v},\\
  \vec{v} &= 0\cdot\vec{u} + 1\cdot\vec{v}.
\end{align*}

Therefore, the column representations of $\vec{u}$ and $\vec{v}$ will be exactly $\vec{u}=\colvec{1;0}$ and $\vec{v}=\colvec{0;1}$. This means that the column vectors $\colvec{1;0}$ and $\colvec{0;1}$ can represent \textbf{any} two (non-zero) vectors we wish! So remember: you must always be sure with which basis vectors you are working. Otherwise, mistakes are bound to happen and calculations might make no sense.

\tbw{more examples?}

Since we need two real numbers to express any such 2-dimensional vector as a column, we call the set of all such vectors $\Rs\times\Rs$, or simply $\Rs[2]$.

\subsection{Vector operations, norm and the zero vector in column form}
Previously we saw how to scale and add vectors (the latter using the parallogram method). Let us now see how we perform these operations using the column representation of vectors. We will use the same basis vectors as before:

\begin{center}
  \begin{tikzpicture}
    % !! Use same basis vectors and v from before?.. !!
    \draw[vector, xred] (O) -- ++(e1) node[midway, above] {$\vec{e}_{1}$};
    \draw[vector, xblue] (3,0) -- ++(e2) node[midway, right] {$\vec{e}_{2}$};
  \end{tikzpicture}
\end{center}

We start with the vector $\xp{\vec{v}}$ from before, i.e. $\xp{3\cdot\vec{v}}=\xr{\vec{e}_{1}}+\xb{2\cdot\vec{e}_{2}}$ and $\xp{\vec{v}}=\xtwocolvec{3}{2}$:

\begin{center}
  \begin{tikzpicture}
    \coordinate (v) at ($3*(e1)+2*(e2)$);
    \draw[vector, xpurple] (O) -- ++(v) node[midway, left] {$\vec{v}$};
    \draw[vector, xred, dashed] (O) -- ++(e1) node[midway, below] {$\vec{e}_{1}$};
    \draw[vector, xred, dashed] (e1) -- ++(e1) node[midway, below] {$\vec{e}_{1}$};
    \draw[vector, xred, dashed] ($2*(e1)$) -- ++(e1) node[midway, below] {$\vec{e}_{1}$};
    \draw[vector, xblue, dashed] ($3*(e1)$) -- ++(e2) node[midway, right] {$\vec{e}_{2}$};
    \draw[vector, xblue, dashed] ($3*(e1)+(e2)$) -- ++(e2) node[midway, right] {$\vec{e}_{2}$};
    \fill (O) circle (0.05);
  \end{tikzpicture}
\end{center}

What then would be the components of, say, $\frac{1}{2}\xp{\vec{v}}$? First we scale $\xp{\vec{v}}$ by $\xp{\frac{1}{2}}$: remember, this just means ``squeezing'' the vector to $\xp{\frac{1}{2}}$ its former length, keeping it pointing in the same direction:

\begin{center}
  \begin{tikzpicture}
    \draw[vector, xpurple] (O) -- ++(v) node[midway, above left] {$\vec{v}$};
    \draw[vector, xpurple] (5,0) -- ++($0.5*(v)$) node[midway, above left] {$\frac{1}{2}\vec{v}$};
    \draw[->, thick] (2.5,1) -- ++(1.5,0) node[midway, above] {$\frac{1}{2}\times$};
  \end{tikzpicture}
\end{center}

Now we use the basis vectors $\xr{\vec{e}_{1}}$ and $\xb{\vec{e}_{2}}$ to get the column representation of $\xp{\frac{1}{2}\vec{v}}$:
\begin{center}
  \begin{tikzpicture}
    \draw[vector, xpurple] (O) -- ++($0.5*(v)$) node[midway, above left] {$\frac{1}{2}\vec{v}$};
    \draw[vector, xred, dashed] (O) -- ++(e1) node[midway, below] {$\vec{e}_{1}$};
    \draw[vector, xred, dashed] (e1) -- ++($0.5*(e1)$) node[midway, below] {$\frac{1}{2}\vec{e}_{1}$};
    \draw[vector, xblue, dashed] ($1.5*(e1)$) -- ++(e2) node[midway, above right] {$\vec{e}_{2}$};
  \end{tikzpicture}
\end{center}

We get that $\xp{\frac{1}{2}\vec{v}}=\xr{1.5\vec{e}_{1}}+\xb{1\vec{e}_{2}}$, i.e. $\xp{\frac{1}{2}\vec{v}}=\xtwocolvec{1.5}{1}=\xtwocolvec{\frac{1}{2}\cdot3}{\frac{1}{2}\cdot2}$ - i.e. scaling $\xp{\vec{v}}$ by $\xp{\frac{1}{2}}$ simply multiplied both its components by $\xp{\frac{1}{2}}$.

This is in fact true for any vector and any scalar, using any coordinate system: scaling a vector by a scalar $\alpha$ results in multiplying each of its components by $\alpha$:
\begin{equation}
  \vec{u} = \xtwocolvec{u^{1}}{u^{2}}\quad\Longrightarrow\quad \alpha\vec{u} = \xtwocolvec{\alpha\cdot u^{1}}{\alpha\cdot u^{2}}.
  \label{eq:column_representation_scaling}
\end{equation}

Now let's add two vectors together: $\xp{\vec{v}}-\xdg{\vec{w}}$ (the same $\xdg{\vec{w}}$ from \autoref{example:more vectors same basis}), again using the same basis vectors. We saw that $\xp{\vec{v}}=\xtwocolvec{3}{2}$ and $\xdg{\vec{w}}=\xtwocolvec{-3}{1.5}$. Using the parallelogram method their addition is the following vector $\vec{a}$:

\begin{center}
  \begin{tikzpicture}
    \draw[vector, xpurple] (O) -- ++(v) node[midway, above left] {$\vec{v}$};
    \draw[vector, xdarkgreen] (v) -- ++(w) node[midway, above] {$\vec{w}$};
    \draw[vector] (O) -- ++($(v)+(w)$) node[midway, left, xshift=-5pt] {$\vec{a}=\xp{\vec{v}}+\xdg{\vec{w}}$};
  \end{tikzpicture}
\end{center}

Now we calculate the components of $\vec{a}$ in the basis $\xr{\vec{e}_{1}},\xb{\vec{e}_{2}}$:

\begin{center}
  \begin{tikzpicture}
    \draw[vector] (O) -- ++($(v)+(w)$) node[midway, left, xshift=-5pt] {$\vec{a}$};
    \draw[vector, xblue, dashed] (O) -- ++(e2) node[midway, above right] {$\vec{e}_{2}$};
    \draw[vector, xblue, dashed] (e2) -- ++(e2) node[midway, above right] {$\vec{e}_{2}$};
    \draw[vector, xblue, dashed] ($2*(e2)$) -- ++(e2) node[midway, above right] {$\vec{e}_{2}$};
    \draw[vector, xblue, dashed] ($3*(e2)$) -- ++($0.5*(e2)$) node[midway, above right] {$\frac{1}{2}\vec{e}_{2}$};
  \end{tikzpicture}
\end{center}

We see that $\vec{a}=\xr{0\cdot\vec{e}_{1}}+\xb{3.5\vec{e}_{2}}=\xtwocolvec{0}{3.5} = \xtwocolvec{3+(-3)}{2+1.5}$, so the column representation of the addition of two vectors is simply the addition of their components. This is always true:
\begin{equation}
  \vec{u}=\xtwocolvec{u^{1}}{u^{2}},\ \vec{v}=\xtwocolvec{v^{1}}{v^{2}}\quad\Longrightarrow\quad\vec{u}+\vec{v}=\xtwocolvec{u^{1}+v^{1}}{u^{2}+v^{2}}.
  \label{eq:column_representation_addition}
\end{equation}

\begin{note}{Scaling and addition using different coordinate systems}{}
This is only true if the column representation of the two vectors is in the same coordinate system (i.e. using the same basis vectors). Adding two column vectors in different coordinate systems (i.e. using different basis vectors) requires changing one of the column representation to the basis of the other one.
\end{note}

Negating a vector can be thought of as multiplying it by the scalar $\alpha=-1$. Therefore, given a vector $\vec{v}=\xtwocolvec{v^{1}}{v^{2}}$, its opposite would be
\begin{equation}
  -\vec{v}=\xtwocolvec{-v^{1}}{-v^{2}}.
  \label{eq:column_representation_opposite}
\end{equation}

And finally, since $\vec{0}=\vec{v}+\left(-\vec{v}\right)$ for any $\vec{v}$, we get that
\begin{equation}
  \vec{0} = \xtwocolvec{v^{1}}{v^{2}}+\left(-1\xtwocolvec{v^{1}}{v^{2}}\right) = \xtwocolvec{v^{1}}{v^{2}}+\xtwocolvec{-v^{1}}{-v^{2}} = \xtwocolvec{v^{1}-v^{1}}{v^{2}-v^{2}} = \xtwocolvec{0}{0}.
  \label{eq:column_representation_zero_vec}
\end{equation}

\begin{summary}{Column representation of vectors}{}
  Scaling and adding vectors using their column representation is rather simple:
  \begin{enumerate}
    \item Scaling a vector by a scalar $\alpha$ is done by multiplying each of the vector's components by $\alpha$: $\alpha\xtwocolvec{v^{1}}{v^{2}}=\xtwocolvec{\alpha v^{1}}{\alpha v^{2}}$.
    \item Adding two vectors is done by adding their respective components: $\xtwocolvec{u^{1}}{u^{2}}+\xtwocolvec{v^{1}}{v^{2}}=\xtwocolvec{u^{1}+v^{1}}{u^{2}+v^{2}}$.
    In general, we say that using the column representation, vector scaling and addition is done \emph{component-wise}.
    \item Using the above operations we get that the opposite of a vector $\xtwocolvec{v^{1}}{v^{2}}$ is $\xtwocolvec{-v^{1}}{-v^{2}}$.
    \item We also get that the zero vector is always $\vec{0}=\xtwocolvec{0}{0}$.

  \end{enumerate}
\end{summary}

\subsection{Cartesian coordinates and the standard basis set}
We can place vectors on a two-dimensional Cartesian coordinate system, such that their origin coincide with the axis-origin (the point $\bm{O}=(0,0)$). We then mark the point where its head lies as $\bm{p}=(p_{x},p_{y})$:

\begin{center}
  \begin{tikzpicture}[]
    \pgfmathsetmacro{\px}{2}
    \pgfmathsetmacro{\py}{1}
    \begin{axis}[
      vector plane,
      width=7cm, height=7cm,
      xmin=-1, xmax=3,
      ymin=-1, ymax=3,
      xtick={1,...,3},
      xticklabels={},
      yticklabels={},
      extra x ticks={\px},
      extra x tick labels={$p_{x}$},
      extra y ticks={\py},
      extra y tick labels={$p_{y}$},
    ]
    \coordinate (vec2d) at (\px,\py);
    \fill[gray] (vec2d) circle (0.05) node[above] {$\bm{p}=(p_{x},p_{y})$};
    \draw[gray, dashed] (vec2d) -- (\px,0);
    \draw[gray, dashed] (vec2d) -- (0,\py);
    \draw[vector, xorange] (0,0) -- (vec2d) node[midway, above] {$\vec{u}$};
    \fill (0,0) circle (0.05) node[] (O2) {};
    \end{axis}
    \node[below left of=O2, xshift=-10pt] (O2txt) {$\bm{O}=(0,0)$};
    \draw[-stealth, thick, densely dotted] (O2txt) to[out=90, in=225, looseness=1.0] (O2);
  \end{tikzpicture}
\end{center}

This arrangement has a fitting basis set, which we call the \emph{standard basis set of $\Rs[2]$}:
\begin{enumerate}
  \item $\xr{\hat{x}}$: a vector of unit length in the direction of the $x$-axis, i.e. from the origin of the axes to point $(1,0)$.
  \item $\xb{\hat{y}}$: a vector of unit length in the direction of the $y$-axis, i.e. from the origin of the axes to point $(0,1)$.
\end{enumerate}

\begin{center}
  \begin{tikzpicture}[]
    \begin{axis}[
      vector plane,
      width=7cm, height=7cm,
      xmin=-1, xmax=3,
      ymin=-1, ymax=3,
      xtick={1,...,3},
      xticklabels={},
      yticklabels={},
      extra x ticks={1},
      extra x tick labels={(1,0)},
      extra y ticks={1},
      extra y tick labels={(0,1)},
    ]
    \coordinate (vec2d) at (2,1);
    \draw[vector, xorange] (0,0) -- (vec2d) node[midway, above] {$\vec{u}$};
    \draw[vector, xred] (0,0) -- (1,0) node[midway, below] {$\hat{x}$};
    \draw[vector, xblue] (0,0) -- (0,1) node[midway, left] {$\hat{y}$};
    \fill (0,0) circle (0.05) node[] (O2) {};
    \end{axis}
  \end{tikzpicture}
\end{center}

Using the standard basis set, the first component of any vector is simply $p_{x}$ and the second component is $p_{y}$, i.e. $\xo{\vec{u}}=\xtwocolvec{p_{x}}{p_{y}}$. For example, we see that the vector $\xo{\vec{u}}$ in the above figure has the column representation $\xo{\vec{u}}=\xtwocolvec{2}{1}$.

\begin{example}{Vectors on the Cartesian plane}{}
  Some more vectors on the 2-dimensions Cartesian plane and their column representation using the standard basis (each grid line is one unit in size):

  \begin{center}
    \begin{tikzpicture}[]
      \begin{axis}[
        vector plane,
        width=7cm, height=7cm,
        xmin=-3, xmax=3,
        ymin=-3, ymax=3,
        xtick={-3,...,3},
        ytick={-3,...,3},
        xticklabels={},
        yticklabels={},
        major grid style={draw=gray!25, densely dotted},
      ]
      \draw[vector, xred] (0,0) -- (1,0) node[midway, below] {$\hat{x}$};
      \draw[vector, xblue] (0,0) -- (0,1) node[midway, left] {$\hat{y}$};
      \draw[vector, xpurple] (0,0) -- (2,3) node[anchor=north west, pos=1.05, black] {$\xtwocolvec{2}{3}$};
      \draw[vector, xdarkgreen] (0,0) -- (-2,1) node[anchor=south, pos=1.05, black] {$\xtwocolvec{-2}{1}$};
      \draw[vector, xorange] (0,0) -- (-2,-2) node[anchor=east, pos=1.05, black] {$\xtwocolvec{-2}{-2}$};
      \draw[vector, gray] (0,0) -- (1,-2) node[anchor=west, pos=1.05, black] {$\xtwocolvec{1}{-2}$};
      \fill (0,0) circle (0.05) node[] (O2) {};
      \end{axis}
    \end{tikzpicture}
  \end{center}
\end{example}

It's common to call $\xr{u^{1}}$ simply $\xr{x}$, and $\xb{u^{y}}$ simply $\xb{y}$, and therefore the components of some vector $\xo{\vec{u}}$ are called $\xr{u^{x}}$ and $\xb{u^{y}}$, respectively.

\subsection{Polar coordinates}
Using the 2-dimensional Cartesian plane we can define an alternative coordinate system for vectors: notice that each vector $\xdg{\vec{u}}=\xtwocolvec{u^{x}}{u^{y}}$ defines a right triangle together with the $x$-axis; The side on the $x$-axis and the side parallel to the $y$-axis are then $\xr{u^{x}}$ and $\xb{u^{y}}$, respectively:

\begin{center}
  \begin{tikzpicture}[every node/.style={font=\Large}]
    \pgfmathsetmacro{\px}{2}
    \pgfmathsetmacro{\py}{3}
    \pgfmathsetmacro{\th}{atan(\py/\px)}
    \begin{axis}[
      vector plane,
      width=7cm, height=7cm,
      xmin=-1, xmax=3,
      ymin=-1, ymax=3,
      xtick={1,...,3},
      xticklabels={},
      yticklabels={},
    ]
    \coordinate (vec2d) at (\px,\py);
    \draw[gray, dashed] (vec2d) -- (\px,0);
    \draw[vector, xdarkgreen] (0,0) -- (vec2d) node[midway, above left] {$\vec{u}$};
    \draw[thick] (1.75,0) -- (1.75,0.25) -- (2,0.25);
		\draw[xred, very thick, decorate, decoration={brace, amplitude=3pt, raise=3pt, mirror}] (0,0) -- (\px,0) node[midway, below , yshift=-5pt]{$u^{x}$};
		\draw[xblue, very thick, decorate, decoration={brace, amplitude=3pt, raise=3pt, mirror}] (\px,0) -- (\px,\py) node[midway, right, xshift=5pt]{$u^{y}$};
    \draw[xpurple, very thick] (0.75,0) arc (0:\th:0.75);
    \draw[xpurple, arcnode={3mm}{$\theta$}] (0.75,0) arc (0:\th:0.75);
    \fill (0,0) circle (0.05) node[] (O2) {};
    \end{axis}
  \end{tikzpicture}
\end{center}

Due to the Pythagorean theorem we know that the norm of $\xdg{\vec{u}}$ is
\begin{equation}
  \xdg{\norm{u}} = \sqrt{\left(\xr{u^{x}}\right)^{2} + \left(\xb{u^{y}}\right)^{2}},
  \label{eq:pythagorean_vec_norm}
\end{equation}
and due to trigonometry, the angle $\xp{\theta}$ between $\xdg{\vec{u}}$ and the $x$-axis is
\begin{equation}
  \xp{\theta} = \arctan\left(\frac{\xb{u^{y}}}{\xr{u^{x}}}\right).
  \label{eq:label}
\end{equation}

The alternative coordinate system, called the \emph{polar coordinate system}, uses a tuple of numbers: $\left(\xg{r},\xp{\theta}\right)$ - where $\xdg{r}=\xdg{\norm{u}}$. The conversion from the coordinates $\left(\xdg{r},\xp{\theta}\right)$ to the column form $\xtwocolvec{u^{x}}{u^{y}}$ is therefore
\begin{align*}
  \xr{u^{x}} &= \xdg{r}\cos(\xp{\theta}),\\\nonumber
  \xb{u^{y}} &= \xdg{r}\sin(\xp{\theta}).
\end{align*}
(cf. \autoref{eq:basic_trig_rearrange})

\tbw{examples of transitioning between coords sys}

\tbw{example: turtle Joe's travels}

\tbw{example: $n$ people pulling on rope, force equals 0 when arranged in a regular polygon. Use Lagrange's trigonometric identities to prove}

\subsection{$\Rs[3]$ and beyond}
\subsection{Linear combinations and subspaces}
\subsection{The scalar product}
\subsection{The cross product}
\subsection{The Gram-Schmidt process}
\subsection{Normal vectors}
\subsection{Usage examples}
